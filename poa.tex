\documentclass{article}

\usepackage[utf8]{inputenc}
\usepackage{amsmath}
\usepackage{amssymb}
\usepackage{amsthm}
\usepackage{amsfonts}
\usepackage{physics}
\usepackage{enumerate}
\usepackage[margin=1in]{geometry}

\title{Summer Of Science Plan Of Action }
\author{Name: Pradipta Parag Bora \\ Mentor: Thariq Shanavas \\ Topic: Quantum Computation and Quantum Information Theory}
\date{April 2020}

\begin{document}
\maketitle
\section{Objective of the Report}
I intend to build my understanding of Quantum Computing and Quantum Information Theory so that I can finally come up with a report that summarises key theorems and results in Quantum Computing. I will mostly start by introducing quantum mechanics and computer science in a rigorous fashion and I will then move onto qubits and the use of quantum phenomenon in computation.
\section{Weekly Breakup of Topics}

Week 1: Dirac Notation, Linear algebraic treatment of Quantum Mechanics.  Introduction to Computer Science: Turing Machines and Computational Classes. Stern Gerlach experiment. \\
Week 2: Quantum Measurement Theory, Qubits, Bloch Representation of Qubits, Universal Quantum Gates and simple Quantum Circuits. \\
Week 3: Mixed Quantum States, Composing Quantum Systems using Tensor products, partial traces. \\
Week 4: Superdense coding, Quantum Teleportation \\ 
Week 5: Kraus operators, Deutsch Algorithm, Quantum Fourier Transform \\
Week 6: Quantum Search algorithms: Grover's algorithm. Shor's algorithm \\
Week 7 onwards: Physical realisations of Quantum systems, Quantum Information Theory

Please note that the above is tentative and the exact weekly coverage may change depending upon the complexity of the topic.

\section{References}
As per the guidance of my mentor I will mostly be following the book \textbf{Quantum Computation and Quantum Information} by \textbf{Nielsen and Chuang}.
For additional help in Quantum Mechanics, I will be consulting \textbf{Principles of Quantum Mechanics} by \textbf{Ramamurti Shankar} and \textbf{Modern Quantum Mechanics} by \textbf{J.J Sakurai}.


\end{document}